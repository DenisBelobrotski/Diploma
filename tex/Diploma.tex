\documentclass[12pt]{article}
\usepackage[russian]{babel}
\usepackage{indentfirst}
\usepackage{mathtools}
\usepackage[left=2cm, right=2cm, top=2cm, bottom=2cm, bindingoffset=0cm]
{geometry}
\usepackage{graphicx}

\begin{document}

    \begin{titlepage}

        \large
        \begin{center}
            МИНИСТЕРСТВО ОБРАЗОВАНИЯ РЕСПУБЛИКИ БЕЛАРУСЬ  БЕЛОРУССКИЙ ГОСУДАРСТВЕННЫЙ УНИВЕРСИТЕТ
            ФАКУЛЬТЕТ ПРИКЛАДНОЙ МАТЕМАТИКИ И ИНФОРМАТИКИ
        \end{center}

        \vspace{1cm}

        \large
        \begin{center}
            \textbf{Численное исследование диффузии частиц в магнитной жидкости в неоднородном магнитном поле}
        \end{center}

        \vspace{1cm}

        \begin{center}
            Отчёт по преддипломной практике
        \end{center}

        \vfill

        \begin{flushright}

            Белоброцкого Дениса Витальевича\\
            студента 4 курса,\\
            специальность\\
            «вычислительная математика»\\
            Научный руководитель:\\
            кандидат физико-математических наук,\\
            доцент Полевиков В. К.\\

        \end{flushright}

        \vspace{1cm}

        \large
        \begin{center}
            \textbf{Минск, 2019}
        \end{center}

    \end{titlepage}

\newpage

\tableofcontents

\newpage

\section{Введение}

    Благодаря своей способности к пондеромоторному взаимодействию с внешним магнитным полем магнитные 
    жидкости не только спровоцировали развитие нового направления в механике жидкости, но и стали новым 
    технологическим материалом, который нашел широкое применение в технике [1–4]. Магнитная жидкость 
    представляет собой стабильную коллоидную суспензию ферромагнитных частиц в жидкости-носителе 
    (масло, вода, биосовместимая жидкость). Размер частиц порядка $ 10^{-8} $ м, и они находятся в состоянии 
    броуновского движения в жидкости-носителе. Благодаря тому, что частицы обладают магнитными свойствами, 
    в магнитной жидкости происходит не только броуновское движение, но и диффузионный процесс 
    магнитофореза [1,5]. Этот процесс диффузии становится значительным, когда магнитная жидкость находится 
    под воздействием сильного градиента магнитного поля. 
    \par
    Предметом настоящего исследования является классическая феррогидростатическая задача о двусвязных 
    равновесных формах магнитной жидкости, расположенной на горизонтальной пластине вокруг вертикального 
    цилиндрического проводника с постоянным током [1,2,6]. Осесимметричные формы свободной поверхности, 
    которые реализуются под воздействием магнитного поля проводника, являются предпочтительными для 
    математической модели из-за структуры магнитного поля. Учитывая линейный закон намагниченности и 
    пренебрегая скачком капиллярного давления на поверхности, задача была решена аналитически, см. [1,2]. 
    Численное решение более детальной задачи, учитывающей как капиллярный скачок, так и (нелинейный) 
    закон намагниченности Ланжевена, реализовано в [6]. Однако следует подчеркнуть, что как простейшие 
    теоретические модели, изученные в [1,2], так и более продвинутые в [6], основаны на предположении 
    об однородности магнитной жидкости, т.е. влиянии магнитофореза ферромагнитных частиц в жидкости. 
    Целью данной работы является исследование влияния диффузии магнитных частиц на равновесные осесимметричные 
    формы поверхности свободной магнитной жидкости. Поскольку величина намагниченности жидкости прямо 
    пропорциональна концентрации частиц в объеме жидкости [1,2,7,8], которая определяется структурой 
    магнитного поля, диффузионный эффект, как ожидается, станет заметным при сильно неоднородном магнитном поле.

\section{Математическая модель}

\subsection{Основные уравнения}

    Предполагая, что магнитные частицы имеют сферическую форму и одинаковый размер, массообмен магнитных частиц 
    в магнитной жидкости можно описать уравнением [7,8]:

    \begin{equation}
        \frac{\partial C}{\partial t} + \mathbf{v} \cdot \nabla C = D \nabla \cdot (\nabla C - CL(\xi H) \nabla (\xi H) - C \eta \mathbf{g})
    \end{equation}

\end{document}
